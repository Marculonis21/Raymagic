\documentclass[a4paper, 12pt]{article}
\usepackage[T1]{fontenc}
\usepackage[utf8]{inputenc}
\usepackage{booktabs}
\usepackage{titling}
\usepackage{titlesec}
\usepackage{amssymb}
\usepackage{pifont}
\usepackage{graphicx}
\graphicspath{ {../../} }

\usepackage{hyperref}
\hypersetup{
    colorlinks=true,
    linkcolor=black,
    urlcolor=cyan,
}
\urlstyle{same}

\renewcommand{\contentsname}{Obsah}
\renewcommand{\thesection}{\Roman{section}}
\renewcommand{\thesubsection}{\roman{subsection}}

\titleformat{\section}
{\Large\bfseries}
{\thesection}
{0.5em}
{}


\titleformat{\subsection}
{\large\bfseries}
{\thesubsection.}
{0.5em}
{}

\title{
        \vspace{1in}
        \rule{\linewidth}{0.5pt}
		\usefont{OT1}{bch}{b}{n}
        \huge Uživatelská dokumentace \\Raymagic\\
        \vspace{-10pt}
        \rule{\linewidth}{1pt}
}
\author{
		\normalfont\normalsize
        Marek Bečvář\\[-3pt]
}
\date{}



\begin{document}
\maketitle 
\newpage

\tableofcontents
\newpage

\section{O programu} 
\paragraph{}
Raymagic je interaktivní ukázka ze hry Portal vytvořená na vlastním
vykreslovacím prostředí. 

V této ukázce může uživatel odehrát řadu úrovní, 
které byly pro tento projekt vytvořeny, a vytvářet a upravovat si vlastní mapy a objektz pomocí speciálního
jazyka, vyvinutého pro tento projekt. 

\subsection{Zdroje}
V programu je využito fontu \textbf{Computer Modern} ze stránky\\
\url{https://www.1001fonts.com/cmu-font.html}. 
Font je zpřístupněn pod licencí: \emph{SIL Open Font License (OFL)}, otevřený
pro osobní i komerční užití. \\\small{(\emph{Celé znění licence je přiložené ve
složce Content/Fonts, ze které je font načítán.})}


\section{Spuštění programu}
\paragraph{C\#}
Projekt je vytvořen v C\# .NET Core 6 s využitím frameworku Monogame verze 3.8.
\url{https://www.monogame.net}. Projekt byl vytvářen na platformě Linux. Pro
uživatele jsou dostupné nerozlišitelné spustitelné verze pro platformy Linux i
Windows.

\subsection{Spuštění}
\paragraph{}
Pro spuštění programu není potřeba žádných dalších kroků, kromě spuštění
přiloženého spustitelného souboru.

\subsection{Doporučené systémové zdroje}
Aplikace velmi aktivně využívá všech dostupných vláken procesoru, proto pro
bezproblémový chod doporučuji využít CPU, které tyto potřeby uspokojí. Pro
načítání úrovní na začátku hry a mezi úrovněmi potřeba minimálně 4 GB RAM.

\section{Herní prostředí}
\paragraph{Spuštění}
Pro spuštění programu je potřeba pouze spustit přiložený spustitelný soubor.
Hráč se následně octne na základní obrazovce (hlavním menu), běžící v konzoli. 

\subsection{Startovní nabídka}
\paragraph{}
Toto menu představuje rozcestník jednotlivých funkcí aplikace. V menu se hráč
pohybuje pomocí šipek na klávesnici a svoji volbu potvrzuje klávesou ENTER.
Vystoupení z vnořeného menu je dosaženo klávesou ESCAPE.

\paragraph{GAMEMODE}
Pod tlačítkem GAMEMODE se nachází menu se seznamem úrovní kampaně, kterou hráč
má možnost v rámci tohoto projektu projít. Hráč si zde volí, v jaké úrovni chce
svůj průchod začít. Ve spodní části menu se pak nachází jméno právě vybrané
mapy.

\paragraph{SANDBOX}
Tlačítko základního menu SANDBOX uživatele umožní volit ze všech map, ať už
zkompilovaných předem, nebo ve složce \url{Maps/GameMaps} najitých souborů s
příponou \textbf{.map}, popisující ve vlastním čitelném jazyce úrovně do hry.

% \begin{center}
%    \includegraphics[width=0.9\linewidth]{img3.png}
% \end{center}

\paragraph{CONTROLS}
Popis všech ovládacích prvků.

\paragraph{ABOUT RAYMAGIC}
Krátký popis projekt.

\subsection{Hra}
Po načtení zvolené mapy, ať už z \textbf{GAMEMODE} nebo \textbf{SANDBOX} módu,
se hráč objeví ve zvolené mapě a může hru ovládat dle kláves představených v
menu \textbf{CONTROLS}.


\section{Editor map}
\paragraph{} 
Součástí projektu je i editor map a objektů, který byl vytvořený jako ulehčení
pro vytváření a upravování map a návrh 3D interaktivních objektů, se kterými se
hráč ve setkává. Ve finální verzi je většina map ve hře vytvořená a fungující
přímo z editoru (což na hru nemá žádný dopad).

\subsection{Soubor .map}
\paragraph{}
Mapy, které jsou vytvářené s editorem lze jednoduše najít ve složce
\url{Maps/GameMaps}, podle přípony \textbf{.map}. Jedná se o vlastní příponu
souboru, která ale pouze označuje textový soubor, který aplikace vyhledává a za
běhu s ním pracuje.

\subsection{Práce s editorem}
\paragraph{}
Aby mohl uživatel s editorem pracovat, musí si nejprve vytvořit vlastní .map
soubor (pro první pokusy doporučeno kopírovat již existující mapy s WIP flagem
a pouze změnit \emph{name} v config bloku na vlastní jméno). 

Dále je potřeba vytvořenou mapu načíst, což jde jednoduše skrz \textbf{SANDBOX}
menu, kam bude záznam o mapě automaticky po restartování aplikace přidán.
Po načtení mapy ještě editor neběží, ale můžeme ho zapnout pomocí klávesy
\textbf{F12}. Pokud vše proběhlo správně, na levé straně u informačních výpisů
naběhla nová informační kolonka s nápisem "\emph{Compiling done}".

V tuto chvíli se soubor s informacemi o mapě každou půl vteřinu kontroluje a
pokud v něm uložíme změny do správných bloků, změny se okamžitě promítnou do
naší mapy. 

\subsection{Pravidla jazyka editoru}
\paragraph{}
Editor podporuje vlastní zadávací jazyk, vytvořený konkrétně pro tuto aplikaci.
Pro správné fungování editoru je potřeba dodržovat pravidla tohoto jazyka. Při
porušení pravidel se mapa nepřeloží, nebo při aktivních úpravách vypíše
chybovou hlášku do informační kolonky.

\paragraph{Rozdělení}
Kód se dělí na bloky, do kterých můžeme zapisovat další výrazy.
Povolené bloky: 
\begin{itemize}
    \item \textbf{config} \color{red}*\normalcolor 
    \item \textbf{static} \color{red}*\normalcolor 
    \item \textbf{dynamic} \color{red}*\normalcolor \color{blue}*\normalcolor
    \item \textbf{lights} \color{red}*\normalcolor \color{blue}*\normalcolor
    \item \textbf{physics} \color{blue}*\normalcolor
    \item \textbf{interactable}
    \item \textbf{portalable}
\end{itemize}
Bloky označené \color{red}*\normalcolor jsou povinné. Bloky označené
\color{blue}*\normalcolor jsou editovatelné za běhu programu.

\paragraph{Help}
Pro pomoc při psaní můžeme při kompilaci vložit do libovolného z povolených
bloků \textbf{help}, který v daném místě hned kompilaci přeruší a vypíše do
informační kolonky rozsáhlý seznam, který popisuje pravidla a využití všech
podporovaných funkcí jazyka. Klasický \textbf{help} je nejobecnější seznam.
Pokud uživatel potřebuje bližší informace o nějakém z bloků, nebo o pravidlech 
nějakého z objektů, může obdržet koncentrovanější informace na dané téma pomocí
\\\textbf{help: <zvolené téma>} (\emph{př. help: config, help: box, help: light}).

\paragraph{Limitace - indoor, outdoor}
Pokud v interaktivním bloku vytvoříme dveře, se specifickým jménem
\textbf{indoor} nebo \textbf{outdoor}, při dalším načtení mapy se do mapy
automaticky vloží načítací mapa připojená na vstupní/výstupní dveře mapy.
Mapa, která má automaticky připojené načítací mapy, již nemůže být upravovaná
za běhu programu!

\section{Závěr}
Projekt byl vytvoření jako záverečná semestrální práce pro předměty
\\\emph{NPRG035 Programování C\#} a {NPRG038 Pokročilé programování C\#} -
zimní a letní semestr 2021/22 - Matfyz UK.
\end{document}
